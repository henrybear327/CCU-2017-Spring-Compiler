%!TEX TS-program = xelatex
%!TEX encoding = UTF-8 Unicode

% sudo easy_install Pygments
% Add -shell-escape to compile flag

% Options -> Editor -> Indentation mode -> replace tab by space!! -> avoids ^^I issue in code

\documentclass[12pt, A4]{article}
\usepackage[utf8]{inputenc}
\usepackage[linguistics]{forest}

%%%%%% Font setting %%%%%%

\usepackage{fontspec} %加這個就可以設定字體 

\usepackage{xeCJK} %讓中英文字體分開設置

%\setCJKmainfont{Noto Sans CJK TC} %設定中文為系統上的字型,而英文不去更動,使用原TeX字型
\setCJKmainfont{STKaiti}

\XeTeXlinebreaklocale "zh" %這兩行一定要加,中文才能自動換行
\XeTeXlinebreakskip = 0pt plus 1pt 

%%%%%% Coding style %%%%%%
\usepackage{minted}
\usemintedstyle{friendly}
\setminted[c]{
    linenos=true,
    autogobble=true,
}

\usepackage{hyperref}

%%%%%% 文件正式開始 %%%%%%

\title{Compiler \\ Assignment 5 \\ Top-Down Parsing}
\author{403410033 \ 資工三 \ 曾俊宏}
\date{\today}

\begin{document}
	
	\maketitle
	\newpage

	\section{Question 1}
	
	\subsection*{a) First sets and follow sets}
	
	\begin{itemize}
		\item First set: 
            \begin{itemize}
                \item first(A) = \{a, b, c, d, e, $\epsilon$\}
                \item first(B) = \{a, b, $\epsilon$\}
                \item first(C) = \{c, $\epsilon$\}
                \item first(D) = \{d, $\epsilon$\}
            \end{itemize}
		\item Follow set:
            \begin{itemize}
                \item follow(A) = \{\$\}
                \item follow(B) = \{c, \$\}
                \item follow(C) = \{d, \$\}
                \item follow(D) = \{e\}
            \end{itemize}
	\end{itemize}
	
    \newpage
	\subsection*{b) Procedures of recursive-decent parser}
	
	\begin{itemize}
        \begin{minted}{c}
        const int a = 1, b = 2, c = 3, d = 4, e = 5;
        int token = lexer();
        void match(int t)
        {
            if (token == t)
                token = lexer();
            else
                error();
        }
        \end{minted}
        
		\item A: 
			\begin{minted}{c}
            void A() {
                switch (token) {
                    case a:
                    case b:
                    case c:
                        B();
                        C();
                        break;
                    case d:
                    case e:
                        D();
                        match(e);
                        break;
                   default:
                        error();
                }
            }
			\end{minted}
        \newpage
		\item B:
            \begin{minted}{c}
            void B() {
                switch (token) {
                    case a:
                        match(a);
                        B();
                        break;
                    case b:
                        match(b);
                        break;
                    default:
                        error();
                }
            }
            \end{minted}
		\item C:
            \begin{minted}{c}
            void C() {
                switch (token) {
                    case c:
                        match(c);
                        C();
                        match(d);
                        break;
                    default:
                        error();
                }
            }
            \end{minted}
        \newpage
		\item D:
            \begin{minted}{c}
            void D() {
                switch (token) {
                    case d:
                        match(d);
                        D();
                        break;
                    default:
                        error();
                }
            }
            \end{minted}
	\end{itemize}
	
    \newpage
	\subsection*{c) Parsing table of table-driven predictive parser}
	
	\begin{tabular}{|c|c|c|c|c|}
		\hline 
		   & A & B & C & D \\ 
		\hline 
		a  & A $\rightarrow$ BC & B $\rightarrow$ aB &  &  \\ 
		\hline 
		b  & A $\rightarrow$ BC & B $\rightarrow$ b  &  &  \\ 
		\hline 
		c  & A $\rightarrow$ BC & & C $\rightarrow$ cCd &  \\ 
		\hline 
		d  & A $\rightarrow$ De & & & D $\rightarrow$ dD \\ 
		\hline 
		e  & A $\rightarrow$ De & & &  \\ 
		\hline 
		\$ &  &  &  &  \\ 
		\hline 
	\end{tabular} 
	
\end{document}