%!TEX TS-program = xelatex
%!TEX encoding = UTF-8 Unicode

% sudo easy_install Pygments
% Add -shell-escape to compile flag

\documentclass[12pt, A4]{article}
\usepackage[utf8]{inputenc}
\usepackage[linguistics]{forest}

%%%%%% Font setting %%%%%%

\usepackage{fontspec} %加這個就可以設定字體 

\usepackage{xeCJK} %讓中英文字體分開設置

\setCJKmainfont{Noto Sans CJK TC} %設定中文為系統上的字型,而英文不去更動,使用原TeX字型

\XeTeXlinebreaklocale "zh" %這兩行一定要加,中文才能自動換行
\XeTeXlinebreakskip = 0pt plus 1pt 

%%%%%% Coding style %%%%%%
\usepackage{minted}
\usemintedstyle{friendly}

\usepackage{hyperref}

%%%%%% 文件正式開始 %%%%%%

\title{Compiler \\ Assignment 5 \\ Top-Down Parsing}
\author{403410033 \ 資工三 \ 曾俊宏}
\date{\today}

\begin{document}
	
	\maketitle
	\newpage

	\section{Question 1}
	
	\subsection*{a) First sets and follow sets}
	
	\begin{itemize}
		\item First set: 
		\item Follow set:
	\end{itemize}
	
	\subsection*{b) Procedures of recursive-decent parser}
	
	\begin{itemize}
		\item A: 
			\begin{minted}[
			linenos=true,
			autogobble=true,
			]{c}
			int main() {
				printf("hello, world");
				return 0;
			}
			\end{minted}
		\item B:
		\item C:
		\item D:
	\end{itemize}
	
	\subsection*{c) Parsing table of table-driven predictive parser}
	
	\begin{tabular}{|c|c|c|c|c|}
		\hline 
		   & A & B & C & D \\ 
		\hline 
		a  &  &  &  &  \\ 
		\hline 
		b  &  &  &  &  \\ 
		\hline 
		c  &  &  &  &  \\ 
		\hline 
		d  &  &  &  &  \\ 
		\hline 
		e  &  &  &  &  \\ 
		\hline 
		\$ &  &  &  &  \\ 
		\hline 
	\end{tabular} 
	
\end{document}